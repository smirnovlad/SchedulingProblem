\section{Введение}

\subsection{Постановка задачи}
Имеется множество работ $J$ и множество машин $M$. Также задана функция $p : J \times M \rightarrow \mathbb{R}_{+}$. Значение $p_{ij}$ означает время выполнения $i$-й работы на $j$-й машине. Требуется построить распределение работ по машинам так, чтобы все работы были выполнены и чтобы конечное время выполнения всех работ было минимально. Иначе говоря, требуется найти функцию $x : J \times M \rightarrow \{0, 1\}$ такую, что:

$$\max_{j \in M} \sum_{i} x_{ij}p_{ij} \rightarrow min$$
$$s.t. \sum_{j \in M} x_{ij} = 1$$

\subsection{Формализация задачи}
В нотации Рональда Грэма поставленная задача обозначается через $P||C_{max}$, где $P$ обозначает идентичных исполнителей и $C_{max}$ -- целевую функцию, которую мы минимизируем. Пусть $|M| = m, |J| = n$. Каждая машина в каждый момент времени решает не более чем одну задачу (работу). Как только машина начинает выполнять $i$-ю работу, через время $p_{i}$ она её завершает и становится готовой выполнять следующую. Обозначим через $J_{j}$ множество работ, назначенных на $j$-ю машину. Таким образом, расписание задаётся набором подмножеств множества данных работ $\{J_{1}, J_{2},\ldots, J_{m}\}$. Обозначим через $$C(J_{j}) = \sum_{i \in J_{j}} p_{i}$$ нагрузку $j$-й машины в соответствии с расписанием. Тогда наша целевая функция определяется как $$C_{max} = \max_{1 \leq j \leq m} C(J_{j}).$$

\noindent
Далее автор использует термины \textit{задача} и \textit{работа} как равносильные. Будем рассматривать частный случай задачи, когда производительности всех машин одинаковы, то есть $p_{ij} = p_{i}$.