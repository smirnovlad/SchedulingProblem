\section{$\bf{NP}$-полнота задачи}

Поставленная задача может быть переформулирована в проблему принятия решения следующим образом:

\textit{Пусть в дополнение к предыдущим входным данным даётся некоторое положительное число $K$. Существует ли назначение задач на машины, такое что нагрузка на каждую машину не превосходит $K$?}

Докажем, что задача об оптимальном расписании для случая идентичных машин является $\bf{NP}$-полной, путём сведения её к задаче об упаковке ($\sf{Bin}$ $\sf{packing}$ $\sf{problem}$). $\bf{NP}$-полнота задачи об упаковке доказывается преобразованием её из задачи о разделении ($\sf{Partition}$ $\sf{problem}$) (см. [2]). Оригинальную идею доказательства автор данной статьи нашёл в документе [1].

Формулировка задача об упаковке:

\textit{Дан некоторый конечный набор элементов $U$. Вес каждого элемента определяется функцией $s: U \rightarrow \mathbb{Z}_{+}$. Даны два положительных целых числа $B$ и $l$. Существует ли упаковка элементов множества $U$ в $l$ множеств $U_{1}, U_{2}, \ldots, U_{l}$, такая что суммарный вес каждого из полученных множеств не превосходит $B$?}

\begin{theorem}
    Поставленная задача $\bf{NP}$-полна.
\end{theorem}

\begin{proof}
    $P||C_{max} \in \bf{NP}$, поскольку недетерминированный алгоритм будет за полиномиальное время проверять, удовлетворяет ли некоторое назначение работ на машины ограничению на суммарное время выполнения $K$. Сведём нашу задачу к задаче об упаковке: $U := J$, $s(u) := p_{i}$ ($i$-я работа -- элемент $u$ множества $U$), $l := |M|$ и $B := K$. Таким образом, $\bf{NP}$-полнота задачи об оптимальном расписании для случая идентичных машин доказана.
\end{proof}